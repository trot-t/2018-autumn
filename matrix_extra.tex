% https://tex.stackexchange.com/questions/67254/stretching-text-vertically
% https://tex.stackexchange.com/questions/18157/rotating-a-letter

%\documentclass[a4paper,12pt]{scrartcl}
\documentclass[a4paper,12pt]{article}
\usepackage{mathtext}
\usepackage{cmap}
\usepackage[utf8]{inputenc}
\usepackage[english, russian]{babel}
%\usepackage[left=1cm,right=1cm,top=1cm,bottom=2cm]{geometry}
\usepackage{tikz}
\usepackage{dsfont}
\usepackage{graphicx}
%\usepackage{amsmath}
%\usepackage{amssymb}
%\usepackage{stackengine}
%\usepackage{scalerel}

\begin{document}

\newcommand{\vectorcol}[2]{%
	\mathrel{\raisebox{-0.25cm}{\scalebox{0.75}[3.1]{#1}$_#2$}}%
}

\underline{{\bfЗадание 1.}} собственные числа и собственные вектора матрицы $\mathds{A}$ удовлетворяют уравнению:

\begin{equation}
\mathds{A}\cdot\vec{V}_i = \lambda_i \vec{V}_i 
\end{equation}
где $\vec{V}_i$ -- вектор-столбец, соответствующий собственному числу $ \lambda_i$.


\underline{{\bf Пояснение к заданию 2.}}

Умножим матрицу  $\mathds{A}$ на матрицу состоящую из вектор-столбцов собственных векторов
\begin{equation}
%\mathds{A}\cdot\raisebox{-1 cm}{\scalebox{0.75}[3]{V}_1} \scalebox{0.75}[3]{V}_2 \scalebox{0.75}[3]{V}_3
\mathds{A}\cdot\left(\raisebox{-0.25cm}{\scalebox{0.75}[3.1]{V}$_1$} \vectorcol{V}{2} \vectorcol{V}{3}\right)
=
\left(\lambda_1\!\!\vectorcol{V}{1}\:\lambda_2\!\!\vectorcol{V}{2}\:\lambda_3\!\!\vectorcol{V}{3}\right)
	\label{uravneniea}
\end{equation}

Возьмём вектора с тильдами 
$\widetilde{\vectorcol{V}{1}}$, $\widetilde{\vectorcol{V}{2}}$ и $\widetilde{\vectorcol{V}{3}}$
такие что скалярные произведения векторов $(\widetilde{\vec{V}}_i \cdot \vec{V}_k)$ 
записанные в виде ``строка на столбец'' удовлетворяют условию:

\begin{equation}
	(\widetilde{\vec{V}}_i \cdot \vec{V}_k) \;= \;
\rotatebox{90}{\raisebox{-0.25cm}{\scalebox{0.75}[3.1]{$\widetilde{V}$}$_{\!\!i}$}}\;\cdot \vectorcol{V}{k}
=
\left\{\begin{array}{l}0, \textcyrillic{ если }i\neq k \\1, \textcyrillic{ если }i=k  \end{array}\right.
\label{Kroneker}
\end{equation}

Составим матрицу строки которой состоят из координат векторов $\widetilde{\vec{V}}_i$: 
\begin{equation}
\left(
\begin{array}{l}
	\rotatebox{90}{\raisebox{-0.25cm}{\scalebox{0.75}[3.1]{$\widetilde{V}$}$_{\!\!1}$}} \\
	\rotatebox{90}{\raisebox{-0.25cm}{\scalebox{0.75}[3.1]{$\widetilde{V}$}$_{\!\!2}$}} \\
	\rotatebox{90}{\raisebox{-0.25cm}{\scalebox{0.75}[3.1]{$\widetilde{V}$}$_{\!\!3}$}}
\end{array}
\right)
	\label{reverse}
\end{equation}

Из условий (\ref{Kroneker}) очевидно, что 
\begin{equation}
\left(
\begin{array}{l}
        \rotatebox{90}{\raisebox{-0.25cm}{\scalebox{0.75}[3.1]{$\widetilde{V}$}$_{\!\!1}$}} \\
        \rotatebox{90}{\raisebox{-0.25cm}{\scalebox{0.75}[3.1]{$\widetilde{V}$}$_{\!\!2}$}} \\
        \rotatebox{90}{\raisebox{-0.25cm}{\scalebox{0.75}[3.1]{$\widetilde{V}$}$_{\!\!3}$}}
\end{array}
\right)
\cdot
\left(\vectorcol{V}{1}\:\vectorcol{V}{2}\:\vectorcol{V}{3}\right) =
\left(
\begin{array}{ccc}
        1 \\
        & 1 \\
        && 1
\end{array}
\right)
	\label{equalone}
\end{equation}

Произведение матриц равно единичной матрице. отсюда следует, что матрица $
\left(
\begin{array}{l}
        \rotatebox{90}{\raisebox{-0.25cm}{\scalebox{0.75}[3.1]{$\widetilde{V}$}$_{\!\!1}$}} \\
        \rotatebox{90}{\raisebox{-0.25cm}{\scalebox{0.75}[3.1]{$\widetilde{V}$}$_{\!\!2}$}} \\
        \rotatebox{90}{\raisebox{-0.25cm}{\scalebox{0.75}[3.1]{$\widetilde{V}$}$_{\!\!3}$}}
\end{array}
\right)
$ есть обратная матрица к матрице $\left(\vectorcol{V}{1} \vectorcol{V}{2} \vectorcol{V}{3}\right)$:

$$
\left(
\begin{array}{l}
        \rotatebox{90}{\raisebox{-0.25cm}{\scalebox{0.75}[3.1]{$\widetilde{V}$}$_{\!\!1}$}} \\
        \rotatebox{90}{\raisebox{-0.25cm}{\scalebox{0.75}[3.1]{$\widetilde{V}$}$_{\!\!2}$}} \\
        \rotatebox{90}{\raisebox{-0.25cm}{\scalebox{0.75}[3.1]{$\widetilde{V}$}$_{\!\!3}$}}
\end{array}
\right) =
\left(\vectorcol{V}{1} \vectorcol{V}{2} \vectorcol{V}{3}\right)^{-1}
$$

Из за того, что матрица (\ref{reverse}) обратная к матрице составленной из собственных векторов-столбцов
их можно переставлять местами:

$$
\left(\vectorcol{V}{1}\:\vectorcol{V}{2}\:\vectorcol{V}{3}\right)\cdot
\left(
\begin{array}{l}
        \rotatebox{90}{\raisebox{-0.25cm}{\scalebox{0.75}[3.1]{$\widetilde{V}$}$_{\!\!1}$}} \\
        \rotatebox{90}{\raisebox{-0.25cm}{\scalebox{0.75}[3.1]{$\widetilde{V}$}$_{\!\!2}$}} \\
        \rotatebox{90}{\raisebox{-0.25cm}{\scalebox{0.75}[3.1]{$\widetilde{V}$}$_{\!\!3}$}}
\end{array}
\right) =
\left(
\begin{array}{ccc}
        1 \\
        & 1 \\
        && 1
\end{array}
\right)
$$

Умножая обратную матрицу слева на левую и правую части уравнения (\ref{uravneniea}) получаем

\begin{equation}
\left(
\begin{array}{l}
        \rotatebox{90}{\raisebox{-0.25cm}{\scalebox{0.75}[3.1]{$\widetilde{V}$}$_{\!\!1}$}} \\
        \rotatebox{90}{\raisebox{-0.25cm}{\scalebox{0.75}[3.1]{$\widetilde{V}$}$_{\!\!2}$}} \\
        \rotatebox{90}{\raisebox{-0.25cm}{\scalebox{0.75}[3.1]{$\widetilde{V}$}$_{\!\!3}$}}
\end{array}
\right)
\cdot 
\mathds{A}\cdot\left(\raisebox{-0.25cm}{\scalebox{0.75}[3.1]{V}$_1$} \vectorcol{V}{2} \vectorcol{V}{3}\right) = 
\left(
\begin{array}{l}
        \rotatebox{90}{\raisebox{-0.25cm}{\scalebox{0.75}[3.1]{$\widetilde{V}$}$_{\!\!1}$}} \\
        \rotatebox{90}{\raisebox{-0.25cm}{\scalebox{0.75}[3.1]{$\widetilde{V}$}$_{\!\!2}$}} \\
        \rotatebox{90}{\raisebox{-0.25cm}{\scalebox{0.75}[3.1]{$\widetilde{V}$}$_{\!\!3}$}}
\end{array}
\right)
\cdot
\left(\lambda_1\!\!\vectorcol{V}{1}\lambda_2\!\!\vectorcol{V}{2}\lambda_3\!\!\vectorcol{V}{3}\right)
=
\left(
\begin{array}{ccc}
	\!\lambda_1\!\\
	&\!\lambda_2\!\\
	&&\!\lambda_3\!
\end{array}
\right)
	\label{transformation}
\end{equation}


Предъявим вектора $\widetilde{\vec{V}}_i$. 

Для этого заметим, что результат векторного произведения векторов перпендикулярен
векторам из которых оно состоит. Это значит что $\widetilde{\vec{V}}_i$ пропорционален векторному произведению
$[\vec{V}_k \times \vec{V}_l]$, где $i\neq k$ и $i\neq l$.


$$
\widetilde{\vec{V}}_i \sim [\vec{V}_k \times \vec{V}_l]
$$

Коэффициент пропорциональности выберем таким чтобы скалярное произведение равнялось 
$1 = (\widetilde{\vec{V}}_i\cdot \vec{V}_i) = 
\rotatebox{90}{\raisebox{-0.25cm}{\scalebox{0.75}[3.1]{$\widetilde{V}$}$_{\!\!i}$}}\;\cdot \vectorcol{V}{i}$ 
когда индексы $i$ совпадают.

\begin{equation}
\vectorcol{V}{i} \;=\;
\widetilde{\vec{V}}_i = \frac{[\vec{V}_k \times \vec{V}_l]}
{ (\vec{V}_i \cdot [\vec{V}_k \times \vec{V}_l])}
	\label{findreverse}
\end{equation}

\underline{{\bfЗадание 2.}}

По уравнению (\ref{findreverse}) найти обратную матрицу матрице
составленной из собвстенных вектор-столбцов.

Доказать, что верны соотношение (\ref{equalone}) и соотношение (\ref{transformation}).  


\end{document}
